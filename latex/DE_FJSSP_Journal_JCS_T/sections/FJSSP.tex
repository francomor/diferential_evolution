\section{The Flexible Job Shop Scheduling Problem} \label{sec:FJSSP}
\vspace{-0.4cm}

FJSSP is an extension of the classic JSSP where a set of machines, not necessarily identical, can process an operation. Consequently, a set of available machines is provided for each operation. The goal is to decide on which machine each operation will be assigned and in what order the operations will be sequenced on each machine so that the makespan is minimized.

FJSSP can be formally described as follows. A set $J = \{J_1, J_2,\ldots, J_n\}$ of independent jobs and a set $U = \{M_1,M_2, \ldots, M_m\}$ of machines are given. A job $J_i$ is broken down by a sequence of $O_{i1},O_{i2}, \ldots, O_{in_i}$ operations to be performed one after the other according to the given sequence. Each operation $O_{ij}$ can be executed on any among a subset $U_{ij} \subseteq U$ of compatible machines. For this reason, the FJSSP can be classified into two categories, partial (P-FJSSP) and total (T-FJSSP)  \cite{kacem2002}. We have partial flexibility whether exists a proper subset $U_{ij} \subset U$, for at least one operation $O_{ij}$, while we have $U_{ij} =U$ for each operation $O_{ij}$ in the case of total flexibility. The processing time of each operation is machine-dependent. 
Pre-emption is not allowed, i.e., each operation must be completed without interruption once started. Furthermore, the machines cannot perform more than one operation at a time. All jobs and machines are available at time 0.

The problem is to assign each operation to an appropriate machine (routing problem),  and  sequence the operations on the machines (sequencing problem) to minimize the makespan ($C_{max}$). This measure is the time needed to complete all the jobs, which is defined as $C_{max} = max_i\{C_i\}$, where $C_i$ is the completion time of the job $J_i$.  Table~\ref{tab:proccesingTime} shows an instance of FJSSP with 3 jobs, 4 machines and 8 operations. The rows and columns correspond to machines and operations, respectively, and the entries of the table are the processing times. 

\begin{table}[tb]
\scriptsize
   \centering
   \caption{Instance Example for FJSSP}
   \label{tab:proccesingTime}%
\begin{tabular}{c|ccc|ccc|cc}
\hline
\multicolumn{1}{l|}{} & \multicolumn{3}{c}{$J_1$} & \multicolumn{3}{c}{$J_2$} & \multicolumn{2}{c}{$J_3$} \\
\hline
\multicolumn{1}{l|}{} & $O_{11}$    & $O_{12}$   & $O_{13}$   & $O_{21}$    & $O_{22}$   & $O_{23}$   & $O_{31}$        & $O_{32}$       \\
\hline
$M_1$                   & -      & 4     & 9     & 2      & 4     & 9     & 8          & 3         \\
$M_2$                   & 6      & 8     & 5     & -      & 6     & -     & 6          & 5         \\
$M_3$                   & 5      & 5     & -     & 1      & 8     & 2     & -          & 8         \\
$M_4$                   & -      & 6     & 7     & 3      & 4     & 2     & 5          & 3        \\
\hline
\end{tabular}
\end{table}
