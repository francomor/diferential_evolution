\section{Introduction}
 
One of the main goals in modern manufacturing systems is efficiency. In this sense, the production scheduling can be considered an important issue. In this field, the job shop scheduling problem (JSSP)~\cite{Pinedo:2008:STA:1477600} is one of the most important and difficult problems. JSSP consists of jobs and machines, where each job consists of operations to be processed in a given order, and each operation is processed by a specific machine. The objective is to find an operation sequence on the machines (schedule) so that the time needed to complete all jobs is minimized. 

The supposition that one machine only processes one type of operation does not reflect the reality of the modern manufacturing systems. The possibility of selecting alternative routes among the machines is useful in production environments where multiple machines, possibly not identical, can perform the same operation (perhaps with different processing times). This flexibility allows the system to absorb changes in work demands or in machine performances. When this factor is included, the problem is known as the Flexible Job Shop Scheduling Problem (FJSSP), becoming in a more realistic production environment and with practical applicability.

The solution of FJSSP involves two decisions: to sequence the operations on the machines and to assign each operation to the appropriate set of machines to minimize the elapsed time to complete all the jobs (makespan or $C_{max}$). These decisions suggest that FJSSP is a complex optimization problem (NP-hard problem~\cite{Garey:1976:CFJ:2782828.2782830}), consequently, the adoption of metaheuristic~\cite{Talbi,Luque:2013:PGA:2564896} has led to better results than classical dispatching or greedy heuristic algorithms~\cite{tang2011,Wang2012917,WANG2017}. Since introduced in 1997 by Storn and Price~\cite{Storn1997}, the Differential Evolution (DE) metaheuristic became very popular among computer scientists and practitioners almost immediately after its original definition. 

DE is a population-based evolutionary algorithm, utilizing real-valued vectors as a population for each generation. DE employs simple mutation and crossover operators to generate new candidate solutions and applies a one-to-one competition scheme to greedily determine whether the new candidate or its parent will survive in the next generation. Besides, the canonical DE requires very few control parameters, a feature that makes it easy to use for the practitioners. Consequently, their success is due to its simplicity and ease implementation, and reliability and high performance. DE algorithms have been applied to many combinatorial optimization problems (\cite{Teoh:2015:DE-CVRP,GRECO20191,Hull:DE-StructuralOpt,Rout2010}, among many others), but as far as we are aware, there is few published research work that describes the use of DE to deal with FJSSP~\cite{YUAN2013246}. 

In~\cite{morero2019}, we design a simple DE to solve FJSSP. As DE was originally devised for solving continuous optimization problems, we adopt a real value representation for FJSSP to make the continuous DE applicable for solving the discrete FJSSP. This implies that algorithmic operations (mutation and recombination) should not be modified or adapted to solve the problem at hand. Another important feature of DE is the little number of parameters to be set when compared to other evolutionary algorithms: with only three parameters, the algorithm behavior can be controlled. However, the success to find good solutions to a problem depends on discovering the correct values of these parameters~\cite{app8101945}. Therefore, we analyze this line to determine the adequate values for these parameters for solving the FJSSP instances. Moreover, a simple local search procedure is embedded in the DE to improve their exploration capacities by solving the problem. Finally, parallelism at algorithmic level~\cite{Talbi} is incorporated into the DE framework to improve its scalability and reduce the computation time.  Resuming, the contributions of this work are:
\begin{itemize}
\item design of a simple DE to solve FJSSP in the real-value search space without affecting the efficacy
\item improvement of the DE performance with a simple and efficient local search procedure
\item proposal of a parallel DE to deal with efficiency issues
\end{itemize}

The experimental methodology we have followed consists of computing the makespan or $C_{max}$ values for the proposed DE and their improvements to solve FJSSP. The obtained results are compared by considering different quality indicators. The current work is an extension of the previous article~\cite{morero2019}, where we extend and reorganize the problem and algorithm description. Moreover, another DE parameter is considered in the experimental study to determine its appropriate value to solve FJSSP. Consequently, both the experimentation and the result analysis are enlarged, also including other metrics to improve the comprehension of the relation between the solution quality and the DE computational effort.

The paper is organized as follows. In Section~\ref{sec:FJSSP}, we introduce the problem formulation and show an illustrative example of input data. In Section~\ref{sec:DE}, we present the basic DE algorithm. In Section~\ref{sec:HDE} we explain our proposal based on DE to solve FJSSP. In the following section, we introduce the experimental design and in Section 6, we evaluate the results. Further, in Section~\ref{sec:compara} we make a comparison between the proposed DE and other solvers present in the literature. Some final remarks and future research directions are given in Section ~\ref{sec:conclu}.
