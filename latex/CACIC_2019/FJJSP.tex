\section{The Flexible Job Shop Scheduling Problem} \label{sec:FJSSP}
\vspace{-0.4cm}
The FJSSP can be formally described as follows. A set $J = \{J_1, J_2, . . . , J_n\}$ of independent jobs and a set $U = \{M_1,M_2, . . . , M_m\}$ of machines are given. A job $J_i$ is broken down by a sequence of $O_{i1},O_{i2}, . . . , O_{in_i}$ operations to be performed one after the other according to the given sequence. Each operation $O_{ij}$ can be executed on any among a subset $U_{ij} \subseteq U$ of compatible machines. We have partial flexibility if there exists a proper subset $U_{ij} \subset U$, for at least one operation $O_{ij}$, while we have $U_{ij} =U$ for each operation $O_{ij}$ in the case of total flexibility. The processing time of each operation is machine-dependent. We denote with $d_{ijk}$ the processing time of operation $O_{ij}$ when executed on machine $M_k$. Pre-emption is not allowed, i.e., each operation must be completed without interruption once started. Furthermore, the machines cannot perform more than one operation at a time. All jobs and machines are available at time 0.

The problem is to assign each operation to an appropriate machine (routing problem), and to sequence the operations on the machines (sequencing problem) in order to minimize the makespan ($C_{max}$). This measure is the time needed to complete all the jobs, which is defined as $C_{max} = max_i\{C_i\}$, where $C_i$ is the completion time of job $J_i$.  Table~\ref{tab:proccesingTime} shows an instance of the FJSSP with 3 jobs, 4 machines and 8 operations. The rows and columns correspond to machines and operations, respectively, and the entries of the table are the processing times. %In this example, we have a partial flexible scenario, an ``$-$'' entry in the table means that a machine cannot execute the corresponding operation, i.e., it does not belong to the subset of compatible machines for that operation. %According to the evolution paradigm used in GAs, we will refer to any solution of FJSP as an individual or chromosome.

% \begin{table}[tb]
% \scrptsize
%   \centering
%   \caption{Instance Example for the FJSSP}
%     \begin{tabular}{ccrrrr}
%    \hline
%           & & $M_1$    & $M_2$    & $M_3$    & $M_4$ \\ \hline
%     \multirow{3}[0]{*}{$J_1$} & $O_{11}$   & --     & 6     & 5     & -- \\
%     & $O_{12}$   & 4     & 8     & 5     & 6 \\
%     & $O_{13}$   & 9     & 5     & -     & 7 \\\hline
%     \multirow{3}[0]{*}{$J_2$} & $O_{21}$   & 2     & -     & 1     & 3 \\
%     & $O_{22}$   & 4     & 6     & 8     & 4 \\
%     & $O_{23}$   & 9     & -     & 2     & 2 \\\hline
%     \multirow{2}[0]{*}{$J_3$} & $O_{31}$   & 8     & 6     & -     & 5 \\
%     & $O_{32}$   & 3     & 5     & 8     & 3 \\ \hline
%     \end{tabular}%
%   \label{tab:proccesingTime}%
% \end{table}%

\begin{table}[tb]
\scriptsize
   \centering
   \caption{Instance Example for the FJSSP}
   \label{tab:proccesingTime}%
\begin{tabular}{c|ccc|ccc|cc}
\hline
\multicolumn{1}{l|}{} & \multicolumn{3}{c}{$J_1$} & \multicolumn{3}{c}{$J_2$} & \multicolumn{2}{c}{$J_3$} \\
\hline
\multicolumn{1}{l|}{} & $O_{11}$    & $O_{12}$   & $O_{13}$   & $O_{21}$    & $O_{22}$   & $O_{23}$   & $O_{31}$        & $O_{32}$       \\
\hline
$M_1$                   & -      & 4     & 9     & 2      & 4     & 9     & 8          & 3         \\
$M_2$                   & 6      & 8     & 5     & -      & 6     & -     & 6          & 5         \\
$M_3$                   & 5      & 5     & -     & 1      & 8     & 2     & -          & 8         \\
$M_4$                   & -      & 6     & 7     & 3      & 4     & 2     & 5          & 3        \\
\hline
\end{tabular}
\end{table}
