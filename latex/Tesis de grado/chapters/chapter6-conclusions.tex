\chapter{Conclusión} 
En este trabajo se presentó un algoritmo \textit{DE} simple para resolver el FJSSP. Se hizo una revisión del material bibliográfico relacionado con técnicas metaheurísticas que ayudan a la comprensión de la tesis. Se describe una introducción a los algoritmos metaheurísticos, luego se presenta una de las posibles calificaciones de las metaheurísticas: en métodos basados en trayectoria y en métodos basados en poblaciones, y por último se explican distintos conceptos de diseños para paralelizar las metaheurísticas. Así también se hizo una revisión de los conceptos de la evolución diferencial, y se presenta el algoritmo propuesto por Storn \& Price, junto a sus 4 etapas: inicialización, mutación, recombinación y selección. Además se muestra el proceso iterativo del algoritmo \textit{DE} en un formato de pseudocódigo.


A continuación, se presenta una descripción del problema a tratar, el diseño de la representación y evaluación de soluciones utilizado, la decodificación de soluciones, el procedimiento de búsqueda local utilizado, la forma en que fue realizada la paralelización para acelerar el cálculo de \textit{DE} y la búsqueda local, y por último el pseudocódigo del algoritmo del \textit{HDE} para FJSSP, el cual sigue el esquema general de \textit{DE}, con la inclusión del proceso de búsqueda local. Además, se agrega como entrada la probabilidad de aplicar búsqueda local ($P_{BL}$), con el fin de permitir al investigador realizar cambios sobre el parámetro.


Finalmente, se describe el diseño experimental llevado a cabo para este trabajo donde fueron seleccionadas una amplia gama de instancias utilizadas en la literatura teniendo en cuenta su complejidad. Además, se analiza la calidad de los resultados considerando los valores de $ C_ {max} $ obtenidos para las distintas mejoras planteadas al algoritmo \textit{DE} descritas anteriormente para resolver las instancias de FJSSP. En primer lugar se estudia  el efecto de usar diferentes valores de $F$ y $Cr$ en el rendimiento del algoritmo, de valores bajos a altos (0.1, 0.5 y 0.9). Se observa que la combinación $F = 0.5 $ y $Cr = 0.9 $ supera a las demás desde el punto de vista de la calidad de los valores de $ C_ {max} $ obtenidos. 
Luego se analiza el impacto de la incorporación del proceso de búsqueda local a través del parámetro $P_{BL}$, donde fueron considerados tres valores diferentes: 0.1, 0.5 y 0.7 (de valores bajos a altos). Los resultados indican que el algoritmo \textit{HDE} con alta probabilidad de aplicar el procedimiento de búsqueda local, puede encontrar las mejores soluciones para el FJSSP. 
Posteriormente se evalúa el efecto en los tiempos al incorporar paralelismo a nivel de iteración. Los resultados muestran que el uso de la paralelización permite acelerar el tiempo de ejecución con respecto al \textit{HDE} secuencial sobre todas las instancias en un valor de 3 veces en promedio.
Por último, se presenta una comparación de los valores de $C_{max}$ obtenidos por el \textit{HDE} con los alcanzados por varios algoritmos competitivos presentes en la literatura destinados a resolver el FJSSP. Los valores de $C_{max}$ obtenidos por el \textit{HDE} son similares a los demás algoritmos para la mayoría de las diez instancias. Como consecuencia, el algoritmo \textit{HDE} propuesto en este trabajo ofrece buenas soluciones a este problema NP-duro de una manera eficiente.


En lo personal, me agradó la actitud de investigar y explorar en lo “desconocido” y en hacer cosas no imaginadas en un inicio, acudiendo al aprendizaje autodidacta con libros, papers, preguntando y solicitando consejos a mi directora y al grupo LISI, e incluso aprendiendo del modo ensayo/error en el proyecto desarrollado.


Aprendí que es muy diferente el estudio en las aulas, al de guiarse de un libro, y aún mas al de investigar académicamente, buscando nuevas propuestas para resolver problemas que no son triviales, expandiendo las fronteras del conocimiento, reflexionando críticamente y adaptándose a los formatos académicos.


Por último, reconocer que mi tesis no habría sido posible sin el apoyo de mi familia, amigos, y en especial a mi directora Carolina como así también a todo el grupo de investigación, que me sumaron a su equipo con una gran amabilidad y cordialidad desde el día uno. A todos ellos que me apoyaron, me dieron ánimos, conversamos, criticaron, etc, muchas gracias.
